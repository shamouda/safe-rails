
\section{Feral Mechanisms in Rails}
\label{sec:rails-cc}

As we discussed in Section~\ref{sec:motivation}, Rails services user
requests entirely independently, with the database acting as a point
of rendezvous for concurrent operations. Given Rails's design goals of
maintaining application logic at the user level, this appears---on its
face---a somewhat cavalier proposition with respect to application
integrity. In response, Rails has developed a range of
concurrency control strategies that largely operate external to the
database and at the application level, which we term \textit{feral
  concurrency control}.

In this section, we outline four major mechanisms for guarding against
correctness anomalies under concurrent execution in Rails. We
subsequently begin our study of 67 open source applications to
determine which of these mechanisms are used in practice. In the following
section, we will determine which are are sufficient to maintain
correct data---and when they are not.

\subsection{Rails Concurrency Control Mechanisms}

Broadly, Rails contains four mechanisms for concurrency control.

\begin{myenumerate}
\item Rails provides support for \textbf{transactions}. By wrapping a
sequence of operations within a special \texttt{transaction} block,
Rails operations will execute transactionally, backed by an actual
database transaction. The database transaction either runs at the
database's configured default isolation level or, as of Rails 4, can
be configured on a per-transaction
basis.\footnote{https://github.com/rails/rails/commit/392eeecc11a291e406db927a18b75f41b2658253}

\item Rails provides support for both optimistic and pessimistic
\textbf{locking}. Applications invoke pessimistic locks on an Active
Record object by calling its \texttt{lock} method, which invokes a
\texttt{SELECT FOR UPDATE} statement in the database. Optimistic
locking is invoked by declaring a special \texttt{lock\_version} field
in an Active Record model. When a Rails process performs a Model
update, Active Record atomically checks that an object's
\texttt{lock\_version} has not changed since the process last read the
object and, if it has not changed, increments \texttt{lock\_version}
and saves the object.

\item Rails provides support for application-level
\textbf{validations}. Each Model has a set of zero or more
validations, or boolean-valued functions that should be run before the
object is actually saved to the database. These validations ensure,
for example, that particular fields within a record are not null or,
alternatively, are unique within the database. Rails provides a number
of built-in validators but also allows arbitrary user-defined
validations (we discuss actual validations further in subsequent
sections). Validations are performed at model save time, and each
validation declared for a given model is run sequentially within a
database-backed transaction.\footnote{The practice of wrapping
  validations in a transaction dates to the earliest Rails
  commit (albeit, in 2004, transactions were supported via a per-Ruby VM
  global lock: \url{https://github.com/rails/rails/blob/db045dbbf60b53dbe013ef25554fd013baf88134/activerecord/lib/active_record/transactions.rb#L6}). However,
  as late as 2010, updates were only partially protected by transactions
  (\url{https://github.com/rails/rails/commit/f4fbc2c1f943ff11776b2c7c34df6bcbe655a4e5}).} If and only if all validations pass is the
model actually saved to the database.\\[-2mm]

In contrast with the prior two mechanisms, Rails validations are a
prominent feature of the Active Record model (i.e., neither
transactions nor locks are actually discussed in the official ``Rails
Guides''). Per the Rails documentation, validations ``are database
agnostic, cannot be bypassed by end users, and are convenient to test
and maintain.'' In contrast, ``[d]atabase constraints and/or stored procedures make the
validation mechanisms database-dependent and can make testing and
maintenance more difficult.'' As we will see in the next section,
validations are widely used by many actual applications.

\item Rails provides support for application-level
\textbf{associations}. Like validations, associations are a core
concept of the Active Record model. As the name suggests, ``an
association is a connection between two Active Record models,''
effectively acting like a foreign key in a traditional
database. Associations can be declared on one or both sides of a
one-to-one or one-to-many relationship, including transitive
dependencies (via a \texttt{:through} annotation). Declaring an
association (e.g., \texttt{:belongs\_to dept}) produces a special field
for the associated record ID within the model (e.g.,
\texttt{dept\_id}). Coupling an association with an appropriate
validation (e.g., \texttt{:presence}) ensures that the association is
indeed valid (and is, via the validation, backed by a database
transaction). Rails does not provide native support for
database-backed foreign key constraints.
\end{myenumerate}

Overall, these four mechanisms provide a range of options for
developers. The first two are squarely in the realm of traditional
concurrency control. In many cases, the latter two are reminiscent of
common database features such as referential
integrity constraints. However, others, such as UDF-based validations,
are reminiscent of proposals for more exotic, semantic-based
concurrency control primitives. We call these latter two
\textit{feral} strategies for concurrency control. As we show in the next section, the
latter two dominate in terms of developer popularity.

\subsection{Adoption in Practice}

To understand exactly how users were interacting with these
concurrency control mechanisms and determine which deserved more
study, we examined their usage in a portfolio of publicly available
open source applications.

\minihead{Application corpus} We selected 67 open source applications
built using Ruby on Rails and Active Record, representing a variety of
application domains, including eCommerce, customer relationship
management, retail point of sale, conference management, content
management, build management, project management, personal task
tracking, community management and forums, commenting, calendaring,
file sharing, Git hosting, link aggregation, crowdfunding, social
networking, and blogging. We sought projects with substantial
code-bases (average: 26,380 lines of Ruby) multiple contributors
(average: 69.2), and relative popularity (measured according to GitHub
stars) on the site. Table~\ref{table:app-summary} provides an
overview.


\begin{table*}
\scriptsize
\begin{tabular}{{|l}*{11}{l}{l|}}\hline
Name & Description & Authors & LoC Ruby & Commits &
 M & {\scriptsize T} & \scriptsize{L} & \scriptsize{V} &
 \scriptsize{A} & \scriptsize{Stars} &  \tiny{Githash} & \tiny{Last
   commit}\\\hline

Canvas LMS & {\scriptsize{Education}} & 132 & 308,113 & 12,889 & 161 & 46 & 12 & 354 & 839 & 1,251 & {\tiny\texttt{3fb8e69}} & {\tiny 10/16/14}\\
OpenCongress & {\scriptsize{Congress data}} & 15 & 30,040 & 1,884 & 106 & 1 & 0 & 48 & 357 & 124 & {\tiny\texttt{850b602}} & {\tiny 02/11/13}\\
Fedena & {\scriptsize{Education management}} & 4 & 48,359 & 1,471 & 104 & 5 & 0 & 153 & 317 & 262 & {\tiny\texttt{40cafe3}} & {\tiny 01/23/13}\\
Discourse & {\scriptsize{Community discussion}} & 440 & 71,338 & 11,502 & 77 & 41 & 0 & 83 & 268 & 12,233 & {\tiny\texttt{1cf4a0d}} & {\tiny 10/20/14}\\
Spree & {\scriptsize{eCommerce}} & 677 & 46,976 & 14,107 & 72 & 6 & 0 & 92 & 252 & 5,582 & {\tiny\texttt{aa34b3a}} & {\tiny 10/16/14}\\
Sharetribe & {\scriptsize{Content management}} & 35 & 30,357 & 7,140 & 68 & 0 & 0 & 112 & 202 & 127 & {\tiny\texttt{8e0d382}} & {\tiny 10/21/14}\\
ROR Ecommerce & {\scriptsize{eCommerce}} & 19 & 16,732 & 1,604 & 63 & 2 & 3 & 219 & 207 & 857 & {\tiny\texttt{c60a675}} & {\tiny 10/09/14}\\
Diaspora & {\scriptsize{Social network}} & 388 & 31,361 & 14,640 & 63 & 2 & 0 & 66 & 128 & 9,571 & {\tiny\texttt{1913397}} & {\tiny 10/03/14}\\
Redmine & {\scriptsize{Project management}} & 10 & 79,483 & 11,049 & 62 & 11 & 0 & 131 & 157 & 2,264 & {\tiny\texttt{e23d4d9}} & {\tiny 10/19/14}\\
ChiliProject & {\scriptsize{Project management}} & 53 & 64,512 & 5,532 & 61 & 7 & 0 & 118 & 130 & 623 & {\tiny\texttt{984c9ff}} & {\tiny 08/13/13}\\
Spot.us & {\scriptsize{Community reporting}} & 46 & 92,737 & 9,280 & 58 & 0 & 0 & 96 & 165 & 343 & {\tiny\texttt{61b65b6}} & {\tiny 12/02/13}\\
Jobsworth & {\scriptsize{Project management}} & 46 & 24,469 & 7,890 & 55 & 10 & 0 & 86 & 225 & 478 & {\tiny\texttt{3a1f8e1}} & {\tiny 09/12/14}\\
OpenProject & {\scriptsize{Project management}} & 63 & 82,764 & 11,185 & 49 & 8 & 1 & 136 & 227 & 371 & {\tiny\texttt{c1e66af}} & {\tiny 11/21/13}\\
Danbooru & {\scriptsize{Image board}} & 25 & 27,812 & 3,738 & 47 & 9 & 0 & 71 & 114 & 238 & {\tiny\texttt{c082ed1}} & {\tiny 10/17/14}\\
Salor Retail & {\scriptsize{Point of Sale}} & 26 & 18,007 & 2,259 & 44 & 0 & 0 & 81 & 309 & 24 & {\tiny\texttt{00e1839}} & {\tiny 10/07/14}\\
Zena & {\scriptsize{Content management}} & 7 & 55,694 & 2,514 & 44 & 1 & 0 & 12 & 43 & 172 & {\tiny\texttt{79576ac}} & {\tiny 08/18/14}\\
Skyline Content management & {\scriptsize{Content management}} & 7 & 10,241 & 894 & 40 & 5 & 0 & 28 & 89 & 127 & {\tiny\texttt{64b0932}} & {\tiny 12/09/13}\\
Opal & {\scriptsize{Project management}} & 6 & 10,643 & 474 & 38 & 3 & 0 & 42 & 96 & 45 & {\tiny\texttt{11edf34}} & {\tiny 01/09/13}\\
OneBody & {\scriptsize{Church portal}} & 33 & 19,867 & 3,976 & 36 & 3 & 0 & 97 & 140 & 1,041 & {\tiny\texttt{2dfbd4d}} & {\tiny 10/19/14}\\
CommunityEngine & {\scriptsize{Social networking}} & 67 & 13,796 & 1,613 & 35 & 3 & 0 & 92 & 101 & 1,073 & {\tiny\texttt{a4d3ea2}} & {\tiny 10/16/14}\\
Publify & {\scriptsize{Blogging}} & 93 & 16,555 & 5,067 & 35 & 7 & 0 & 33 & 50 & 1,274 & {\tiny\texttt{4acf86e}} & {\tiny 10/20/14}\\
Comas & {\scriptsize{Conference management}} & 5 & 6,893 & 435 & 33 & 6 & 0 & 80 & 45 & 21 & {\tiny\texttt{81c25a4}} & {\tiny 09/09/14}\\
BrowserContent management & {\scriptsize{Content management}} & 56 & 21,011 & 2,503 & 32 & 4 & 0 & 47 & 77 & 1,183 & {\tiny\texttt{d654557}} & {\tiny 09/30/14}\\
RailsCollab & {\scriptsize{Project managment}} & 25 & 8,799 & 865 & 29 & 6 & 0 & 40 & 122 & 262 & {\tiny\texttt{9f6c8c1}} & {\tiny 02/16/12}\\
Insoshi & {\scriptsize{Social network}} & 16 & 118,619 & 1,321 & 28 & 2 & 0 & 63 & 164 & 1,583 & {\tiny\texttt{9976cfe}} & {\tiny 02/24/10}\\
OpenGovernment & {\scriptsize{Government data}} & 15 & 8,906 & 2,231 & 28 & 4 & 0 & 22 & 141 & 160 & {\tiny\texttt{fa80204}} & {\tiny 11/21/13}\\
Tracks & {\scriptsize{Personal productivity}} & 89 & 17,312 & 3,121 & 27 & 2 & 0 & 24 & 43 & 639 & {\tiny\texttt{eb2650c}} & {\tiny 10/02/14}\\
GitLab & {\scriptsize{Code management}} & 672 & 37,671 & 12,319 & 24 & 15 & 0 & 131 & 114 & 14,129 & {\tiny\texttt{72abe9f}} & {\tiny 10/20/14}\\
Brevidy & {\scriptsize{Video sharing}} & 2 & 7,118 & 6 & 24 & 1 & 0 & 74 & 56 & 167 & {\tiny\texttt{d0ddb1a}} & {\tiny 01/18/14}\\
Alchemy & {\scriptsize{Content management}} & 34 & 19,097 & 4,222 & 23 & 2 & 0 & 37 & 40 & 240 & {\tiny\texttt{91d9d08}} & {\tiny 10/20/14}\\
Teambox & {\scriptsize{Project management}} & 48 & 32,252 & 3,155 & 22 & 2 & 0 & 56 & 116 & 1,864 & {\tiny\texttt{62a8b02}} & {\tiny 09/20/11}\\
Fat Free CRM & {\scriptsize{Customer relationship}} & 99 & 20,754 & 4,144 & 21 & 3 & 0 & 39 & 92 & 2,384 & {\tiny\texttt{3dd2c62}} & {\tiny 10/17/14}\\
linuxfr.org & {\scriptsize{FLOSS community}} & 29 & 8,060 & 2,271 & 20 & 1 & 0 & 50 & 50 & 86 & {\tiny\texttt{5d4d6df}} & {\tiny 10/14/14}\\
Squash & {\scriptsize{Bug reporting}} & 28 & 15,663 & 231 & 19 & 6 & 0 & 87 & 62 & 879 & {\tiny\texttt{c217ac1}} & {\tiny 09/15/14}\\
Shoppe & {\scriptsize{eCommerce}} & 14 & 3,115 & 349 & 19 & 1 & 0 & 58 & 34 & 208 & {\tiny\texttt{19e60c8}} & {\tiny 10/18/14}\\
nimbleShop & {\scriptsize{eCommerce}} & 12 & 7,513 & 1,805 & 19 & 0 & 0 & 47 & 34 & 47 & {\tiny\texttt{4254806}} & {\tiny 02/18/13}\\
Piggybak & {\scriptsize{eCommerce}} & 16 & 2,205 & 383 & 17 & 1 & 0 & 51 & 35 & 166 & {\tiny\texttt{2bed094}} & {\tiny 09/10/14}\\
wallgig & {\scriptsize{Wallpaper sharing}} & 6 & 5,541 & 350 & 17 & 1 & 0 & 42 & 45 & 18 & {\tiny\texttt{4424d44}} & {\tiny 03/23/14}\\
Rucksack & {\scriptsize{Collaboration}} & 7 & 5,309 & 445 & 17 & 3 & 0 & 18 & 79 & 169 & {\tiny\texttt{59703d3}} & {\tiny 10/05/13}\\
Calagator & {\scriptsize{Calendaring}} & 48 & 8,877 & 1,766 & 16 & 0 & 0 & 8 & 11 & 196 & {\tiny\texttt{6e5df08}} & {\tiny 10/19/14}\\
Amahi Platform & {\scriptsize{Home media sharing}} & 15 & 6,187 & 577 & 15 & 2 & 0 & 38 & 22 & 65 & {\tiny\texttt{5101c8b}} & {\tiny 08/20/14}\\
Sprint & {\scriptsize{Project management}} & 5 & 3,053 & 71 & 14 & 0 & 0 & 50 & 45 & 247 & {\tiny\texttt{584d887}} & {\tiny 09/17/14}\\
Citizenry & {\scriptsize{Community directory}} & 17 & 7,939 & 512 & 13 & 0 & 0 & 12 & 45 & 138 & {\tiny\texttt{e314fe4}} & {\tiny 04/01/14}\\
Saasy & {\scriptsize{eCommerce}} & 2 & 161,092 & 21 & 12 & 5 & 0 & 41 & 117 & 520 & {\tiny\texttt{4fe610f}} & {\tiny 08/03/09}\\
LovdByLess & {\scriptsize{Social network}} & 17 & 29,639 & 150 & 12 & 0 & 0 & 27 & 41 & 568 & {\tiny\texttt{26e79a7}} & {\tiny 10/09/09}\\
lobste.rs & {\scriptsize{Link sharing}} & 24 & 4,927 & 624 & 12 & 8 & 0 & 20 & 40 & 646 & {\tiny\texttt{b0b9654}} & {\tiny 10/18/14}\\
BucketWise & {\scriptsize{Personal finance}} & 10 & 4,343 & 258 & 12 & 2 & 0 & 11 & 46 & 484 & {\tiny\texttt{5c73f2b}} & {\tiny 06/10/12}\\
Sugar & {\scriptsize{Forum}} & 13 & 7,590 & 1,316 & 11 & 1 & 0 & 20 & 53 & 89 & {\tiny\texttt{49ca79f}} & {\tiny 10/21/14}\\
Comfortable Mexican Sofa & {\scriptsize{Content management}} & 106 & 8,831 & 1,748 & 10 & 0 & 0 & 35 & 26 & 1,523 & {\tiny\texttt{fecef0c}} & {\tiny 10/09/14}\\
Radiant & {\scriptsize{Content management}} & 100 & 15,124 & 2,385 & 9 & 3 & 0 & 26 &
12 & 1,554 & {\tiny\texttt{0c9ef9b}} & {\tiny 10/01/14}\\
Refinery Content management & {\scriptsize{Content management}} & 438 & 10,797 & 9,112 & 9 & 0 & 0 & 16 & 8 & 2,979 & {\tiny\texttt{f4e24ef}} & {\tiny 10/20/14}\\
Forem & {\scriptsize{Forum}} & 106 & 4,632 & 1,409 & 9 & 0 & 0 & 10 & 29 & 1,302 & {\tiny\texttt{519f2de}} & {\tiny 08/14/14}\\
BostonRB & {\scriptsize{Ruby community}} & 40 & 2,128 & 889 & 7 & 0 & 0 & 18 & 12 & 199 & {\tiny\texttt{05fc100}} & {\tiny 10/21/14}\\
Inkwell & {\scriptsize{Social networking}} & 6 & 6,731 & 156 & 7 & 0 & 0 & 4 & 51 & 327 & {\tiny\texttt{d1938d3}} & {\tiny 07/15/14}\\
Boxroom & {\scriptsize{File sharing}} & 9 & 1,924 & 368 & 6 & 0 & 0 & 18 & 12 & 218 & {\tiny\texttt{1e74e06}} & {\tiny 10/18/14}\\
Copycopter & {\scriptsize{Copy writing}} & 9 & 2,267 & 46 & 6 & 1 & 0 & 7 & 14 & 652 & {\tiny\texttt{d3607c4}} & {\tiny 06/28/12}\\
Enki & {\scriptsize{Blogging}} & 29 & 4,584 & 562 & 6 & 1 & 0 & 5 & 7 & 835 & {\tiny\texttt{b793d48}} & {\tiny 12/01/13}\\
Fulcrum & {\scriptsize{Project planning}} & 46 & 3,054 & 637 & 5 & 0 & 0 & 13 & 15 & 1,335 & {\tiny\texttt{8397de2}} & {\tiny 08/20/14}\\
GitLab CI & {\scriptsize{Continuous integration}} & 80 & 3,650 & 870 & 5 & 2 & 0 & 11 & 13 & 1,188 & {\tiny\texttt{7d51134}} & {\tiny 10/17/14}\\
Kandan & {\scriptsize{Persistent chat}} & 56 & 1,533 & 808 & 5 & 0 & 0 & 6 & 8 & 2,249 & {\tiny\texttt{15a8aab}} & {\tiny 10/06/14}\\
Juvia & {\scriptsize{Commenting}} & 8 & 2,280 & 202 & 4 & 3 & 0 & 11 & 8 & 937 & {\tiny\texttt{43a1c48}} & {\tiny 05/09/14}\\
Go vs Go & {\scriptsize{Go board game}} & 2 & 2,317 & 302 & 4 & 0 & 0 & 11 & 9 & 145 & {\tiny\texttt{c8d739d}} & {\tiny 02/21/13}\\
Adopt-a-Hydrant & {\scriptsize{Civics}} & 14 & 14,163 & 1,242 & 3 & 0 & 0 & 11 & 8 & 182 & {\tiny\texttt{5b7ea0e}} & {\tiny 10/21/14}\\
Selfstarter & {\scriptsize{Crowdfunding}} & 23 & 574 & 127 & 3 & 0 & 0 & 1 & 4 & 2,688 & {\tiny\texttt{740075f}} & {\tiny 05/16/14}\\
Heaven & {\scriptsize{Code deployment}} & 19 & 2,083 & 387 & 2 & 0 & 0 & 2 & 2 & 163 & {\tiny\texttt{2d4162e}} & {\tiny 10/21/14}\\
Carter & {\scriptsize{eCommerce}} & 3 & 1,052 & 70 & 2 & 1 & 0 & 0 & 12 & 22 & {\tiny\texttt{60ad49d}} & {\tiny 07/22/14}\\
Obtvse & {\scriptsize{Blogging}} & 27 & 427 & 393 & 1 & 0 & 0 & 3 & 0 & 1,516 & {\tiny\texttt{1542856}} & {\tiny 03/21/13}\\\hline
\textbf{Average:} &  &  & \textbf{26,380.48} & \textbf{2,953.31} &
\textbf{29.09} & \textbf{3.87} & \textbf{0.24} & \textbf{53.00} &
\textbf{96.04} & \textbf{1,272.42} &  & \tiny{\textbf{02/06/14}} \\

\hline
\end{tabular}
\caption{Corpus of applications used in analysis. (M: Models, T:
  Transactions, L: Locks, V: Validations, A: Associations)}
\label{table:app-summary}
\end{table*}

While several of these applications are indeed projects undertaken by
hobbyists, many are either commercially supported (e.g., Canvas LMS,
Discourse, Spree, GitLab) and/or have a large open source community
(e.g., Radiant, Comfortable Mexican Sofa, Diaspora). Undoubtedly, a
larger-scale commercial, closed-source Rails projects such as Twitter,
GitHub, or Airbnb might exhibit different trends than those we observe
here. However, in the open source domain, we believe these
applications represent a diverse selection of use cases.

\minihead{Mechanism usage} We performed a simple analysis of the
applications to determine how each of the concurrency control
mechanisms from each application were used (see Appendix for
methodological details).

Overwhelmingly, applications used validators and associations instead
of transactions or locks (Figure~\ref{fig:usages} and
Table~\ref{table:app-summary}). No application used more transactions
than locks. On average, applications used 0.13 transactions, 1.82
validations, and 3.30 associations per model (with an average of 29.1
models per application).

In the next section, we explore exactly what these validations and
associations are protecting against, but, for now, we note that we
initially found this deviation from traditional concurrency control
somewhat surprising. That is, rather than adopting the use of
traditional transactional updates, Rails application writers chose to
instead declaratively specify correctness criteria and have the ORM
system check the criteria for them instead. It is unclear and even
unlikely that these declarative criteria are a complete specification
of program correctness: undoubtedly, some of these programs contain
errors. However, from the perspective of database usability and
idiomatic web programming patterns, this reliance on application-level
correctness criteria hints at an alternative developer mentality.

\begin{figure}
  \newcommand{\skipht}{\\[-2em]}
\includegraphics[width=\columnwidth]{figs/models-single-bar.pdf}\skipht
\includegraphics[width=\columnwidth]{figs/transactions-single-bar.pdf}\skipht
\includegraphics[width=\columnwidth]{figs/validations-single-bar.pdf}\skipht
\includegraphics[width=\columnwidth]{figs/associations-single-bar.pdf}\skipht
\caption{Use of concurrency control mechanisms in Rails applications}
\label{fig:usages}
\end{figure}

\minihead{Additional metrics} To better illustrate how programmers
were using each of these mechanisms, we report on two additional
analyses.

First, we analyzed the number of models, transactions, validations,
and associations over each project's lifetime. Using each
application's Git logs, we repeated the above analysis at a fixed set
of intervals through the application's history (by commits). At each
interval, we recorded the number of occurrences of each of these
constructs relative to the total number of occurences in the project
in the latest repository we examined. Figure~\ref{fig:historical}
plots the median number of occurences across all projects. Notably, by
the time 50\% of the commits have been entered into the repository,
over 75\% of the models, transactions, validations, and associations
have been written. This hints that the Model layer may be less
volatile than the View and Controller components of the
applications. Moreover, the initial proportion of validations and
associations is higher than the initial proportion of transactions:
the data model appears to stabilize faster than the controller
logic. It is unclear whether the bulk of transaction usage is added in
order to compensate for, say, concurrency violations, or is instead
due to growth in Controller code.

\begin{figure}
\includegraphics[width=\columnwidth]{figs/historical-median.pdf}
\caption{Median use of concurrency control mechanisms over time.}
\label{fig:historical}
\end{figure}

Second, we analyze the distribution of authors to commits compared to
the distribution of authors to validations and associations
authored.\footnote{We chose to analyze commits authored rather than
  lines of code written as git tracks large-scale code refactoring
  commits as an often large set of deletions and
  insertions. Nevertheless, we observed a close correlation between
  lines of code and commits authored.} As Figure~\ref{fig:cdfs} demonstrates, 95\% of all commits
are authored by 42.4\% of authors. However, 95\% of invariants are
authored by only 20.3\% of authors. This is reminiscent of traditional
database schema authorship, where a smaller number of authors modify
the schema than the actual application code.

\begin{figure}
  \newcommand{\skipht}{\\[-2em]}
\includegraphics[width=\columnwidth]{figs/commit-authorship-cdf.pdf}\vspace{-1em}
\includegraphics[width=\columnwidth]{figs/invariant-authorship-cdf.pdf}
\caption{Use of concurrency control mechanisms in Rails applications}
\label{fig:cdfs}
\end{figure}

\subsection{Discussion}

Returning to the Rails design philosphy, the applications we have
encountered do indeed express their logic at the application
layer. There is little actual communication of correctness critera to
the database layer. Part of this is due to limitations within
Rails. As we have mentioned, there is no way to actually declare a
foreign key constraint in Rails without importing additional
third-party modules. However, insofar as Rails is an ``opinionated''
framework encouraging an idiomatic programming style, if our
application corpus is any indication, DHH and his
co-authors appear to have succeeded en masse.

Having observed the relative popularity of these mechanisms, we turn
our attention to the question of their correctness. Specifically, do
these application-level criteria actually enforce the constraints that
they claim to enforce? We restrict ourself to studying the declared
validations and associations for three reasons. First, as we have
seen, these constructs are more widely used in the codebases we have
studied, and, ostensibly, more widely used in the Rails community than
the alternatives. Second, these constructs represent a deviation from
standard concurrency control techniques and are perhaps more likely to
contain errors. Third, while a code analysis of all program logic in
these applications in order to determine whether they are indeed
correct would be instructive, it is likely beyond the capacity of 
automated techniques. We instead opt to check \textit{declared}
constraints rather than latent constraints.
