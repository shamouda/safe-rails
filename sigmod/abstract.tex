
\begin{abstract} The rise of Web 2.0 internet services has led to a range of popular new programming frameworks. In this work, we examine one such framework---Ruby on Rails---and its use and abuse of database concurrency control mechanisms. Namely, Rails programmers eschew the use of traditional tranaction-oriented programming in lieu of application-level mechanisms including declarative validations and associations. We quantitatively analyze the use of these mechanisms in a range of open source projects and determine which actually ensure integrity---and which of these feral, application-level mechanisms may lead to data corruption, demonstrating actual integrity violations when this is the case.  \end{abstract}
