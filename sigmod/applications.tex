
\section{Safety and Correctness: Theory}
\label{sec:apps}

We now turn our attention to understanding which of Rails' feral
validations and associations are actually correct under the execution
model discussed in Section~\ref{sec:deployment}. 

Recall that each of a model's declared validations is run before a model instance is saved to
the database. To correctly enforce a validation requires either that
\textit{i.)} the validations are isolated from one another or
that \textit{ii.)} the validations are somehow ``safe'' to run
concurrently.

Are validations isolated? Given that each sequence of
validations is wrapped within a transaction, under serializable
isolation, validations would appear to execute correctly. However, as
is common in relational database engines~\cite{hat-vldb}, neither
MySQL nor PostgreSQL actually default to serializable isolation, and
instead provide the weaker Read Committed isolation. Under this
isolation level, as we will see shortly, the validations effectively
run concurrently. While Rails 4 does provide support for changing the
isolation level on a per-transaction basis, Rails does not actually
change the isolation level for validations. Similarly, none of the
application code or configurations actually modified the default
isolation level. Moreover, we did not encounter any application
deployment documentation that suggested changing the isolation
level. Although we cannot prove that this is the case, this data
suggests that validations are likely running at Read Committed
isolation.

Does a lack of serializable isolation actually affect these validations? Just
because validations effectively run concurrently does not mean that they are
necessarily incorrect.

% dataset and methodology

% simple static test

% num txns, num invariants

% custom UDFs

% who writes invariants?



% i-confluence results
